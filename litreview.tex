\chapter{Literature Review}\label{C:us}

\section{Are we Ready for a Safer Construction Environment~\cite{SaferConstruction}}
This paper by Yossi Gil and Tali Shragai discusses the cases where a Java program is dependent on subclasses being constructed under the Uniform Identity inheritance model. It discusses the three key stages of object creation and how each of these contributes to the issues surrounding the construction of objects within class hierarchies. These stages are:
\begin{enumerate}
	\item Memory allocation
	\item Preliminary field initialisation
	\item Establishment of invariants
\end{enumerate}
Each of these is dealt with differently across different programming languages. As an example, preliminary field initialisation is approached quite differently in c++ when compared with Java. Java takes the approach of initialising these fields to default values (nulls, zeros and falses) whereas, in the interest of performance, c++ simply leaves these fields with whatever bytes were already present in the memory locations. \newline
Variations between different languages’ implementations of the final stage, the establishment of invariants, are what leads to differing rules surrounding what the developer can and can’t do safely in an object constructor. This is where we find that maintaining a Uniform Identity throughout construction is vital in ensuring that any references to the self which were stored externally during construction remain valid after this process is completed. \newline
We also run into another issue with the changing of the self reference during the construction of an object. During the initialisation of a subclass, it is necessary at some point to initialise the superclass so that it’s fields are defined after construction. If, during the initialisation of the superclass, the self reference is different to that of the subclass, then any calls to overridden methods will execute the superclass's implementation rather than the subclass’s.

\section{Understanding the Shape of Java Software~\cite{ShapeOfJava}}
This paper details an empirical study of a Java corpus to find details about the structure of typical Java programs. The study collected a large set of Java classes and looked at the frequency of occurrence of a variety of common patterns including the ways developers are typically making use of inheritance and composition. As a result of the study it was found that the frequency of several of these patterns exhibited a power-law. \newline
A further interesting finding of the study was a fairly wide variation in the occurrences of some patterns from project to project. This indicates that some architectural decisions may contribute heavily to the patterns employed by developers as the project goes on. This also makes it evident that it will be important, in my own empirical study, to ensure that I have a wide range of Java projects from which to gather my statistics to minimise the bias that would be introduced in a smaller dataset.

\section{Micro Patterns in Java Code~\cite{JavaMicropatterns}}
This paper explores the use of micro patterns found in Java programs. These micro patterns are described as “similar to design patterns, except standing at a lower, closer to the implementation, level of abstraction.”\newline
The patterns I intend to search for as possible examples of forwarding and delegation fit under this definition as each can be expressed as a function over the content of the class.

\section{The Qualitas Corpus: A Curated Collection of Java Code for Empirical Studies~\cite{QualitasCorpus}}
This paper discusses many of the choices behind the construction of the Qualitas Corpus. Notably, the Java language was chosen for a few specific reasons:
\begin{itemize}
	\item Open source Java code is abundant and easy to find. Much more so than c\#, and similarly to c++.
	\item Java code tends to be relatively easier to analyse than many other languages including c++ due to some of the limitations it applies to program structure.
\end{itemize}

\section{What Programmers Do with Inheritance in Java~\cite{InheritanceInJava}}
This paper goes into detail about the use of inheritance in Java projects and the extent to which classes extend others. To aid with this, the paper also contains a formal definitions of a few terms which are relevant to my study, including subtypes, supertypes and downcalls. These definitions are then used to indicate rates of presence in the Qualitas Corpus of a variety of combinations of the patterns. This is achieved by representing the dependencies within the projects as a graph structure and investigating the properties of that graph.

\section{Towards a Metrics Suite for Object Oriented Design~\cite{MetricsSuite}}
Included in this paper are a variety of useful terms for defining measurements of inheritance within programs written in object oriented languages. These include:
\begin{itemize}
	\item \textbf{Depth of Inheritance Tree (DIT)} - A measure of the number of ancestor classes which can potentially affect a given class. This is the depth of the class in the class hierarchy.
	\item \textbf{Number of children (NOC)} - The number of immediate subclasses under a given class in the class hierarchy. This is the number of classes which will, unless explicitly overridden, inherit the methods of the parent class.
	\item \textbf{Coupling between objects (CBO)} - A measure of the non-inheritance relationships a class shares with other classes. This can be seen as a measure of the interdependence of classes in a given program.
\end{itemize}

\section{How Do Java Programs Use Inheritance? An Empirical Study of Inheritance in Java Software~\cite{HowProgramsUseInheritance}}
The authors here explore the use of inheritance in Java programs, primarily in large-scale software development projects, to form a more clear idea of the extent to which particular inheritance patterns are used. The analysis involved over 100,000 classes and interfaces across 90 Java projects. The results of this study show that approximately three quarters of all Java classes in the study had some transitive superclass other than Object in at least half of the examined corpus. \newline
Another contribution of this paper is an explicit discussion of the distinction Java, along with similar languages, makes with regard to it's $extends$ and $implements$ relationships between classes and their superclasses and interfaces respectively. 

\section{Object Inheritance Without Classes~\cite{InheritanceWithoutClasses}}
Tim's paper talks about several different models of object based inheritance and the inherent limitations of both. From this, it becomes evident which Java programs are dependent on the Uniform Identity model which is used to construct instances of Java classes within inheritance hierarchies. Knowing which classes are dependent on this model also informs us about which classes would need to be substantially rethought in order to reimplement them in a language which does not support uniform identity.

\section{Does JavaScript Software Embrace Classes?~\cite{JSClassFinder}}
JavaScript is a useful language to investigate for this study because it provides many examples where programmers use are deliberately using a language built for delegation and object based inheritance to model classical inheritance models. \textit{Does JavaScript Software Embrace Classes} explores the ways in which programmers typically model class inheritance and the ways this can be detected in corpora of JavaScript code. As part of this paper, the researchers also create a tool named JSClassFinder which serves the purpose of identifying both class declaration patterns and class method patterns. The figures returned by this tool can then be analysed to determine the extent to which JavaScript developers are working around the language's inbuilt structures.
