\chapter{Code Patterns}\label{C:bg}

\section{Forms of Code Reuse}
\subsection{Forwarding}
\textit{Forward this.f(...) calls to other.f(...) passing along any necessary information as call parameters} \newline\newline
In Java, this involves searching for patterns where an object’s method does very little work besides forwarding the call to method on another object. This is the simplest form of delegating responsibility to another class and should be independent from any state held by the delegatee, since it needs to be able respond correctly to requests from other delegators without influence from state set in previous requests.\newline 
If the receiver of a forwarded request were to hold state about an object delegating to it then it would likely run into issues if other objects also forward requests to it. Likewise, if the system were re-implemented with a stateful forwarding recipient in a language which supports forwarding as object inheritance then it would run into the same problems when sharing it between parents.\newline
In the following example, a Square object is forwarding responsibility for its area calculation to the SquareAreaCalculator. The SquareAreaCalculator could be shared by many Squares as it holds no state and therefore does not rely on being instantiated as an instance specific and isolated to any given square.

\begin{lstlisting}
class Square{
	int x, y, wd;
	SquareAreaCalculator areaCalculator = new SquareAreaCalculator();
	
	int area(){return areaCalculator.calculate(wd);}
	
	Square(int x, int y, int wd){
		this.x = x; this.y = y; this.wd = wd;
	}
}

class SquareAreaCalculator{
	int calculate(int wd){return wd * wd;}
}
\end{lstlisting}

\subsection{Delegation}
\textit{Forward this.f(...) calls to other.f(...) \textbf{on behalf of this.}
That is, call other.f(...) but have the self reference within that call set to my self reference.}\newline\newline
Examples in Java which would be well suited to a language with native support for delegation are those where the code is effectively forwarding to an object which accepts \textbf{this} as either a constructor parameter or as a parameter to many of its public methods. This indicates that the object being called to is designed to do a lot of work which is dependent on the \textbf{this} reference of another object being passed in. By using a language which supports delegation natively, it would be possible to change the self reference of the delegatee to instead be the self reference of the delegator, removing the need to pass it as a parameter.\newline
In this example, the Square object is delegating responsibility for area calculation to the SquareAreaCalcuator. The SquareAreaCalculator contains a final field to point to the self reference of a single Square object which indicates that the calculator belongs to one instance of Square and always will. In an object delegation model the public final field could be removed, instead opting to have the self reference of the SquareAreaCalculator set to the self reference of the Square object.
\begin{lstlisting}
class Square{
	int x, y, wd;
	SquareAreaCalculator areaCalculator = new SquareAreaCalculator(this);
	
	int area(){return areaCalculator.calculate();}
	
	Square(int x, int y, int wd){
		this.x = x; this.y = y; this.wd = wd;
	}
}

class SquareAreaCalculator{
	private final Square square;
	
	SquareAreaCalculator(Square square){this.square = square;}
	
	int calculate(){return square.wd * square.wd;}
}
\end{lstlisting}

\subsection{Uniform Identity}
\textit{All examples of classical inheritance in Java follow the uniform identity construction model}\newline\newline
Under Uniform Identity, objects are constructed by first going up the object hierarchy setting up fields, then going back down the hierarchy calling initialiser functions. This maintains a single object identity throughout construction of the object.\newline
To find examples supporting the need for Uniform Identity, we must simply look for typical uses of inheritance in Java where a subclass makes some use of functionality from the parent class. Uniform Identity is the most similar object inheritance model to Java’s class inheritance pattern but others can also model the behaviour fairly closely. For example, Merged Identity closely matches the c++ model of class inheritance and, with a few exceptions, most examples of Java class based inheritance could also function in a Merged Identity model. Uniform Identity is the implementation Java’s class based inheritance model encourages so it is expected to be the most common across corpus data. Because of this, Any substantial usage of the previously mentioned examples would indicate that developers are intentionally dismissing Java’s in-built language features as they believe it is possible to produce better code with other patterns.\newline
In this example, the Square class inherits from another class which knows how to calculate the area of a more general case so can also be used to offer the same functionality to the Square. 
\begin{lstlisting}
class Rectangle{
	int x, y, wd, ht;
	
	int area(){return wd * ht;}
	
	Rectangle(int x, int y, int wd, int ht){
		this.x = x; this.y = y; this.wd = wd; this.ht = ht;
	}
}

class Square extends Rectangle{
	Square(int x, int y, int wd){super(x, y, wd, wd);}
}
\end{lstlisting}

\subsection{Java Patterns}
\begin{tabular}{|p{5cm}|p{9cm}|}
	\hline
	\multicolumn{2}{|c|}{Java}                                                                                                                                               \\ \hline
	Forwarding                     & anything name (anything)\{ \newline   return identifier{[}.identifier{]}*.name(anything);\newline \} \newline Where "name" is the same in both places \\ \hline
	Delegation                     & anything name (anything) \{ \newline   return identifier{[}.identifier{]}*.name(this);\newline \}  \newline Where "name" is the same in both places                    \\ \hline
	Constructor Delegation & anything anything = new anything ( this )                                                                                                  \\ \hline
	Inheritance                    & class extends anything                                                                                                                                         \\ \hline
\end{tabular}

\subsection{JavaScript Patterns}
\begin{tabular}{|p{5cm}|p{9cm}|}
	\hline
	\multicolumn{2}{|c|}{JavaScript}                                                                                                                                                                  \\ \hline
	Inheritance 1                  & var a = function ( b ) \{    c . call ( this , d );\}                                                                                      \\ \hline
	Inheritance 2                  & function Bar   ( x , y ) \{    Foo . call ( this , x ) ;\}                                                                                 \\ \hline
	Inheritance 3                  & Foo . prototype = object . create ( Bar . prototype )                                                                                      \\ \hline
	Inheritance 4 - Node.js        & var className = defineClass(...)                                                                                                           \\ \hline
	Inheritance 5 - Node.js        & util.inherits(...)                                                                                                                         \\ \hline
\end{tabular}