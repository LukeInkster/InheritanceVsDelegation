\chapter{Code Patterns}\label{C:bg}
To determine how developers are using delegation and inheritance, the first stage is to define a clear way of identifying the structure of each pattern. This is achieved by defining code patterns which indicate the use of delegation or inheritance. This chapter outlines the patterns which are representative of forwarding, delegation, and Uniform Identity. Lastly, it will briefly cover a few object inheritance models which were not explored in this study.

\section{Forwarding}
\label{sec:forwarding}
Under a forwarding model, calls to \java{this.f(...)} are passed to some \java{other.f(...)}, transferring any necessary information as call parameters.
\newline

In the following example, a \java{Square} object is forwarding responsibility for its area calculation to the \java{SquareAreaCalculator}. The \java{SquareAreaCalculator} could be shared by many \java{Square} objects as it holds no state, and therefore does not rely on being instantiated specifically for one \java{Square}.

\begin{lstlisting}
class Square{
	int x, y, wd;
	SquareAreaCalculator areaCalculator = new SquareAreaCalculator();

	int area(){return areaCalculator.calculate(wd);}

	Square(int x, int y, int wd){
		this.x = x; this.y = y; this.wd = wd;
	}
}

class SquareAreaCalculator{
	int calculate(int wd){return wd * wd;}
}
\end{lstlisting}

In Java, detecting this behaviour involves searching for patterns where an object contains method which does very little work besides forwarding the call to a method on another object. This is the simplest form of transferring responsibility to another class and should be independent from any state held by the forwardee. This is because the forwardee needs to be capable of responding correctly to requests from other forwarders without influence from program state set in previous requests.
\newline

If the receiver of a forwarded request were to hold state about one object delegating to it then it would likely run into issues if other objects also forward requests to it. Likewise, if the system were re-implemented with a stateful forwarding recipient in a language which supports forwarding as object inheritance then it would run into the same problems when sharing it between parents.

\section{Delegation}
\label{sec:delegation}
Delegation is often described as forwarding \java{this.f(...)} calls to \java{other.f(...)} \textbf{on behalf of this.} That is, dispatch the call to \java{other.f(...)} but have the self reference within that call point back to my self reference. This is explained in further detail in Section \ref{inheritanceModels}.
\newline

In this example, the \java{Square} object is delegating responsibility for area calculation to the \java{SquareAreaCalcuator}. The \java{SquareAreaCalculator} contains a final field to point to the self reference of a single \java{Square} object which indicates that the \java{SquareAreaCalculator} belongs to one instance of \java{Square} and always will. In an object delegation model the public final field could be removed, instead opting to have the self reference of the \java{SquareAreaCalculator} set to the self reference of the \java{Square} object.
\begin{lstlisting}
class Square{
	int x, y, wd;
	SquareAreaCalculator areaCalculator = new SquareAreaCalculator(this);

	int area(){return areaCalculator.calculate();}

	Square(int x, int y, int wd){
		this.x = x; this.y = y; this.wd = wd;
	}
}

class SquareAreaCalculator{
	private final Square square;

	SquareAreaCalculator(Square square){this.square = square;}

	int calculate(){return square.wd * square.wd;}
}
\end{lstlisting}

Examples in Java which would be well suited to a language with native support for delegation are those where the code is effectively forwarding to an object which accepts \java{this} as either a constructor parameter or as a parameter to many of its public methods. This indicates that the object being called to is designed to do a lot of work which is dependent on the \java{this} reference of another object being passed in. By using a language which supports delegation natively, it would be possible to change the self reference of the delegatee to instead be the self reference of the delegator, removing the need to pass it as a parameter.

\section{Uniform Identity}
\label{sec:uniformIdentity}
All examples of classical inheritance in Java follow the Uniform Identity construction model as this is built into the language's semantics. Because of this, the analysis undertaken for Uniform Identity will only briefly touch on the use of Uniform Identity by seeking typical uses of inheritance in Java where a subclass makes some use of functionality from the parent class. In addition to this, we seek examples of code which would not function properly under another object inheritance model, for reasons discussed in Section \ref{inheritanceModels}. These potential issues include the presence of constructors which store the self reference outside the class, or which make a local method call which is dispatched to a method which could be overridden.
\newline

Under Uniform Identity, objects are constructed by first going up the object hierarchy setting up fields, then going back down the hierarchy calling initialiser functions. This maintains a single object identity throughout construction of the object~\cite{InheritanceWithoutClasses}. In this example, the \java{Square} class inherits from another class which knows how to calculate the area of a more general figure so can also be used to offer the same functionality to the \java{Square}. Under Uniform Identity, when constructing an instance of the \code{Square} class the \code{Square} type will have its fields initialised to default values first, followed by the \code{Rectangle} supertype. After this, the constructors will run in the opposite order, \code{Rectangle} first then \code{Square} second. Because of this, if the \code{Rectangle}'s constructor calls to a method declared on \cs{Rectangle} but overridden on \cs{Square}, that method will reach an object which has not yet had its constructor called. This is an issue because an object which has not yet had its constructor called will likely be dependent on state which has not been initialised.
\begin{lstlisting}
class Rectangle{
	int x, y, wd, ht;

	int area(){return wd * ht;}

	Rectangle(int x, int y, int wd, int ht){
		this.x = x; this.y = y; this.wd = wd; this.ht = ht;
	}
}

class Square extends Rectangle{
	Square(int x, int y, int wd){super(x, y, wd, wd);}
}
\end{lstlisting}

Uniform Identity is the implementation Java's class based inheritance model encourages, so it is expected to be the most common across corpus data. Because of this, any substantial use of forwarding or delegation would indicate that developers are intentionally dismissing Java's in-built language features as they believe it is possible to produce better code with other patterns.

\section{Other Object Inheritance Models}
\textit{Object Inheritance Without Classes}~\cite{InheritanceWithoutClasses} by Timothy Jones et al. discusses a few other object inheritance models in addition to those covered in this chapter. These models were excluded from this study due to time constraints and the difficulty in attaining corpora for languages with support for the models. Despite this, some of these alternatives offer interesting properties, and vary in terms of their similarity to the models included in this study.

\subsection{Concatenation}
Concatenation was created as an alternative to forwarding and delegation~\cite{InheritanceWithoutClasses}. The aim of concatenation was to offer the power of inheritance while mitigating some of the downsides~\cite{Concatenation}. Rather than storing a reference to the prototype object, as is done in delegation, concatenation simply takes a shallow copy of the state and functionality of the prototype object then stores this information in the new child object. This offers the same behaviour as delegation in most cases, but prevents future changes to the prototype object from affecting the child object, and vice versa.

\subsection{Merged Identity}
Merged Identity is a classical inheritance model similar to Uniform Identity. However, in Merged identity an inheriting object will take on the identity of its parent gradually throughout the construction process~\cite{InheritanceWithoutClasses}. Under this model, the parent object is created first, along with its identity, then that parent object will be mutated by replacing its properties with those of the child object where overrides are declared. This model matches object construction as it occurs in C++.
